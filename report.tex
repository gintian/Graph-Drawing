\documentclass[11pt]{article}
\usepackage[margin=1.25in]{geometry}
\usepackage{graphicx,float}
\usepackage{amsmath,amsthm,amsfonts}
\usepackage{subcaption} % For side-by-side figures
\usepackage{algorithm,algorithmicx,algpseudocode} % For algorithms

% For labelings/struct descriptions
\usepackage{blindtext}
\usepackage{scrextend}
\addtokomafont{labelinglabel}{\sffamily}

\usepackage{listings} % For source code
\usepackage{algorithm,algorithmicx,algpseudocode} % For algorithms

%opening
\title{Force Directed Drawing}
\author{Vincent La}

\begin{document}

% \tableofcontents
\maketitle

\section{Introduction}
Force directed algorithms attempt to draw graphs by relating them to some physical analogy. For example, we may view vertices as steel balls and the edges between them as springs. One of the earlier force directed algorithms, Tutte's Barycenter Algorithm, attempts to place a graph's nodes along it's "center of mass."

\section{Tutte's Barycenter Method}
An early force directed drawing method was Tutte's Barycenter Method. In this method, the force on every vertex is given by 

\[ F(v) = \sum_{(u, v) \in E} (p_u - p_v) \]

Hence, we can ...

\[
\begin{aligned}
    \sum_{(u, v) \in E} (x_u - x_v) &= 0 \\
    \sum_{(u, v) \in E} (y_u - y_v) &= 0 \\
\end{aligned}
\]

Which we may rewrite as

\[
\begin{aligned}
    \deg{(v)}x_v - \sum_{u \in N_1(v)} x_u &= \sum_{w \in N_0(v)} x^*_w \\
    \deg{(v)}y_v - \sum_{u \in N_1(v)} y_u &= \sum_{w \in N_0(v)} y^*_w
\end{aligned}
\]

These equations are linear, and the resulting matrix is diagonally dominant (see Example 1.1). This is because the diagonal consists of vertex degrees, while the other entries $a_{ij}$ are either -1's (if $x_i$ and $x_j$ are neighbors) or 0's if they aren't.

\subsection{Example: Hypercube}
A simple example for which Tutte's method gives aesthetically pleasing results is the hypercube.

\begin{figure}[H]
    \includegraphics[width=\linewidth]{"report/prism_4".pdf}
\end{figure}

In the image below, the hypercube is placed in 500 x 500 pixel grid. The grid is governed by a simple Cartesian coordinate system, where the top left and bottom right corners have coordinates $(-250, 0)$ and $(250, 250)$ respectively. Four vertices are fixed and laid out into a circle of radius 250 centered at the origin. Hence, the bulk of the work performed algorithm is done in placing the center four free vertices. Labeling the free vertices as $x_1, x_2, x_3, x_4$, we may represent the task of laying out the free vertices with this matrix

\[
\begin{bmatrix}
    3 & -1 & 0 & -1 \\
    -1 & 3 & -1 & 0 \\
    0 & -1 & 3 & -1 \\
    -1 & 0 & -1 & 3 \\
\end{bmatrix}
\begin{bmatrix} x_1 \\ x_2 \\ x_3 \\ x_4 \end{bmatrix} =
\begin{bmatrix} 0 \\ 250 \\ 0 \\ -250 \end{bmatrix}
\]

The solution to this matrix is given by $x_1 = x_3 = 0, x_2 = \frac{250}{3}, x_4 = -\frac{250}{3}$.

\subsection{Algorithms}
\subsubsection{Newton-Raphson Iteration}
\begin{algorithm}
    \caption{Barycenter Layout}\label{euclid}
    \begin{algorithmic}[1]       
        % Begin Describing Algorithm   
        \Procedure{barycenter}{$t$}
        \State Layout $n$ fixed vertices in a convex polygon
        \State Construct an (size of free vertices) matrix $A$ as follows
        \[
        A_{i, j} = \begin{cases}
        \deg(v) &{\text{if $i = j$}} \\
        -1      &{\text{if adjacent}} \\
        0       &{\text{otherwise}} \\
        \end{cases}
        \]
        \State Solve $Mx = f_x$ to find the x-coordinates
        \State Solve $My = f_y$ to find the y-coordinates
        \EndProcedure
    \end{algorithmic}
\end{algorithm}

\subsubsection{Linear System}
As seen above, the problem of applying the barycenter method can also been modeled with a linear system.

\begin{algorithm}
    \caption{Barycenter Layout}\label{euclid}
    \begin{algorithmic}[1]       
        % Begin Describing Algorithm   
        \Procedure{barycenter}{$t$}
        \State Layout $n$ fixed vertices in a convex polygon
        \State Construct an (size of free vertices) matrix $A$ as follows
        \[
        A_{i, j} = \begin{cases}
            \deg(v) &{\text{if $i = j$}} \\
            -1      &{\text{if adjacent}} \\
            0       &{\text{otherwise}} \\
        \end{cases}
        \]
        \State Construct an (number of free vertices) vector $x$ as follows
        \[
        x_{i} = \sum{...}
        \]
        \State Solve $Mx = f_x$ to find the x-coordinates
        \State Solve $My = f_y$ to find the y-coordinates
        \EndProcedure
    \end{algorithmic}
\end{algorithm}

\subsection{Prism Graph}
The prism graph denoted $\Pi_{n}$ is constructed by taking the vertices and edges of an $n$-prism. If we look below, the distance between vertices gets increasingly smaller as $n$ increases.

\subsubsection{Symmetry}
Under certain conditions, the barycenter method produces drawings which preserve rotational symmetry.

\paragraph{Theorem: Points along a unit circle form an eigenvector} Consider the linear system for coordinates of the free vertices of the prism graph $\Pi_n$, and then take its corresponding matrix $M$. Now, starting at $(1, 0)$, place $n$ points equally along the perimeter of the unit circle. If we create a vector $x = (x_1, ..., x_n)$, then where $x_i$ is the x-coordinate of the $i^{th}$ unit circle point, then $x$ is an eigenvector of $M$.

\bigskip

Moreover, the same can be said for the y-coordinates and if we write $Mx = \lambda_x x$, $My = \lambda_y y$, it can be shown that $\lambda_x = \lambda_y$.

\begin{proof}
    First, notice that $M = 3I + N$, where $N$ is a matrix composed of all of the $-1's$ in $M$ (and is zero everywhere else). Hence, $N\vec{x}$ is of the form
    \[
        \begin{bmatrix}
            0  & -1  & 0   & ... & 0   & -1 \\
            -1 & \ddots   & \ddots  & 0   & ... & 0 \\
            0  & \ddots  & \ddots   & \ddots  & ... & 0 \\
            0  & ... & \ddots   & \ddots  & \ddots   & 0 \\
            0 & 0   & ... & \ddots   & \ddots  & -1 \\
            -1 & 0   & ... & 0   & -1  & 0 \\
        \end{bmatrix}
        \begin{bmatrix}
            \cos{0} \\ 
            \cos{\frac{2 \pi}{n}} \\
            \dots \\
            \cos{\frac{2 \pi(n-2)}{n}} \\
            \cos{\frac{2 \pi(n-1)}{n}}
        \end{bmatrix}
    \]
    
    By distributivity,
    \[
    Mx = (3I + N)x = 3Ix + Nx
    \]
    
    Hence, showing that $Nx = \lambda x$ is equivalent to showing that the following holds for some $\lambda \in \mathbb{R}$.
    
    \[\begin{cases}
        -\cos{\frac{2 \pi}{n}} - \cos{\frac{2\pi(n-1)}{n}} = \lambda \cos{0} &\text{Equation for the first row} \\
        -\cos{\frac{2 \pi (i - 2)}{n}} - \cos{\frac{2 \pi i}{n}} = \lambda \cos{\frac{2\pi(i-1)}{n}}
        &\text{Equation for the $i^{th}$ row} \\
    \end{cases}\]
    
    Now, the first equation implies that 
    \[\begin{aligned}
        \lambda
        &= -\left[
            \cos{\frac{2\pi}{n}} + \cos{\frac{2\pi n - 2\pi}{n}}   
            \right] \\
        &= -2\left[
            \cos{ \frac{
                2 \pi + 2\pi n - 2\pi    
            }{ 2n } }
            \cos{ \frac{
                2\pi - 2\pi n + 2\pi
                }{ 2n }
            } \right]
        &\text{Using sum-product identity} \\
        &= -2\left[
            \cos{ \frac{2 \pi n}{2n} } \cos{\frac{ 4\pi - 2\pi n }{2n} }
        \right] \\
        &= -2\left[
            \cos{ \pi } \cos{ \frac{ 2\pi}{n} - \pi }
        \right] \\
        &= 2\left[
        \cos{ \pi - \frac{ 2\pi}{n}} \right]
        &\text{cos is an even function} \\
        &= -2\left[ \cos{ - \frac{ 2\pi}{n}} \right]
        = -2\left[ \cos{ \frac{ 2\pi}{n}} \right]
        &\text{Supplementary angles}
   \end{aligned}\]
   
   Now, we just need to show take the equation for the $i^{th}$ row and notice that
   \[
   \begin{aligned}
        -\left[\cos{\frac{2 \pi (i - 2)}{n}} + \cos{\frac{2 \pi i}{n}}\right]
        &= -2\left[
            \cos{ \frac{2\pi(i - 2) + 2\pi i}{2n} }
            \cos{ \frac{2\pi(i - 2) - 2\pi i}{2n} }
        \right]
        &\text{Sum-product identity} \\
        &= -2\left[
            \cos{ \frac{\pi i - 2\pi + \pi i}{n} }
            \cos{ \frac{\pi i + 2\pi - \pi i}{n} }
        \right] \\
        &= -2\left[
            \cos{ \frac{\pi i - 2\pi + \pi i}{n} }
            \cos{ \frac{2\pi}{n} }
        \right] \\
        &= \lambda \cos{ \frac{2\pi(i - 1)}{n} }
   \end{aligned}
   \]
   
   as desired.
    
\end{proof}

\subsubsection{Resolution}
One the the main drawbacks of this algorithm is potentially poor resolution, i.e. the more edges and vertices we add to our graph, the harder it becomes to distinguish the different features of our graph. This is demonstrated best by the prism graph.

\begin{figure}[H]
    \begin{subfigure}{.3\textwidth}
        \includegraphics[width=\linewidth]{"report/prism_4".pdf}
    \end{subfigure}
    \begin{subfigure}{.3\textwidth}
        \includegraphics[width=\linewidth]{"report/prism_5".pdf}
    \end{subfigure}
    \begin{subfigure}{.3\textwidth}
        \includegraphics[width=\linewidth]{"report/prism_6".pdf}
    \end{subfigure}
    \begin{subfigure}{.3\textwidth}
        \includegraphics[width=\linewidth]{"report/prism_7".pdf}
    \end{subfigure}
    \begin{subfigure}{.3\textwidth}
        \includegraphics[width=\linewidth]{"report/prism_8".pdf}
    \end{subfigure}
    \begin{subfigure}{.3\textwidth}
        \includegraphics[width=\linewidth]{"report/prism_9".pdf}
    \end{subfigure}
\caption{$\Pi_4$ through $\Pi_{9}$ as drawn by Tutte's algorithm. Notice that $\Pi_4$ is isomorphic to the hypercube $Q_2$}
\end{figure}

\begin{figure}[H]
    \begin{subfigure}{.5\textwidth}
        \includegraphics[width=\linewidth]{"report/prism_5".pdf}
    \end{subfigure}
    \begin{subfigure}{.5\textwidth}
        \includegraphics[width=\linewidth]{"report/prism_10".pdf}
    \end{subfigure}
    \begin{subfigure}{.5\textwidth}
        \includegraphics[width=\linewidth]{"report/prism_20".pdf}
    \end{subfigure}
    \begin{subfigure}{.5\textwidth}
        \includegraphics[width=\linewidth]{"report/prism_40".pdf}
    \end{subfigure}
    \caption{$\Pi_5, \Pi_{10}, \Pi_{20}$ and $\Pi_{40}$ as drawn by Tutte's algorithm}
\end{figure}

\paragraph{Theorem} For every fixed vertex $u$ in the prism graph $\Pi_{n}$, the distance between it and its adjacent free vertex $v$ is $O(\frac{1}{n})$.

\begin{proof}
    From the theorem above, we know that
    \[ v = u \cdot \frac{1}{3 -2\cos{\frac{2\pi}{n}}} \]
    
    Hence,
    \[
    \text{dist}\left({u, v}\right) = 
    \sqrt{
    % x-coord        
    \left(u_x - u_x \cdot \frac{1}{3 -2\cos{\frac{2\pi}{n}}}
    \right)^2 +
    %
    % y-coord
    \left(u_y - u_y \cdot \frac{1}{3 -2\cos{\frac{2\pi}{n}}}
    \right)^2
    }
    \]
\end{proof}

\end{document}
